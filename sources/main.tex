\documentclass{article}
\usepackage[utf8]{inputenc}

\usepackage[T2A]{fontenc}
\usepackage[utf8]{inputenc}
\usepackage[russian]{babel}

\usepackage{amsmath}

\usepackage{pgfplots}
\usepackage{sagetex}
\usepgfplotslibrary{fillbetween}
\usepgfplotslibrary{polar}
\pgfplotsset{compat=newest}

\usepackage{multienum}
\usepackage{geometry}
\geometry{
    left=1cm,right=1cm,top=2cm,bottom=2cm
}
\newcommand*\diff{\mathop{}\!\mathrm{d}}

\newtheorem{definition}{Определение}
\newtheorem{theorem}{Теорема}

\DeclareMathOperator{\sign}{sign}

\usepackage{hyperref}
\hypersetup{
    colorlinks, citecolor=black, filecolor=black, linkcolor=black, urlcolor=black
}

\title{Высшая математика}
\author{Лисид Лаконский}
\date{April 2023}

\begin{document}
\raggedright

\maketitle

\tableofcontents
\pagebreak

\section{Определенные интегралы и функции многих переменных, вариант №21}

\subsection{Задание №2}

Найти точки экстремума и точки перегиба функции

$$
\Phi(x) = \int\limits_{0}^{x} (4t - t^2) \diff t
$$

\textbf{Решение.}

$\Phi(x) = \int\limits_{0}^{x} (4t - t^2) \diff t = (2t^2 - \frac{t^3}{3}) \bigg|_{0}^{x} = 2x^2 - \frac{x^3}{3}$

$\Phi'(x) = 4x - x^2$, $4x - x^2 = 0 \Longleftrightarrow x (4 - x) = 0 \Longleftrightarrow x_1 = 0, x_2 = 4$ — точки экстремума

Изобразим знаки $\Phi'(x)$ и $\Phi(x)$ на координатной прямой с отмеченными точками $x = 0$, $x = 4$. Найдем точки максимума и минимума: $x_{\text{min}} = 0$, $x_{\text{max}} = 4$

$\Phi(0) = 0$, $\Phi(4) = 32 - \frac{64}{3} = \frac{32}{3}$

\hfill

$\Phi''(x) = f'(x) = 4 - 2x$, $4 - 2x = 0 \Longleftrightarrow x = 2$ — точка перегиба

Изобразим знаки $\Phi''(x)$ и $\Phi(x)$ на координатной прямой с отмеченной точкой $x = 2$. Определим, на каких промежутках график функции вогнут, а на каких выпукл.

$\Phi''(0) = 4$ — график вогнутый, $\Phi''(2) = 0$, $\Phi(x) < 0 \ \forall \ \Phi(x), x > 2$ — график выпуклый

\hfill

\textbf{Ответ.} Точки экстремума: $x = 0$, $x = 4$. Точки перегиба: $x = 2$

\subsection{Задание №3}

Нарисовать область, ограниченную линиями, и вычислить ее площадь

$$
x = -2y^2, \ x = 1 - 3y^2
$$

\textbf{График.}

\begin{center}
\begin{tikzpicture}
\begin{axis}[grid=both,xmax=5,ymax=5,xmin=-5,ymin=-5,samples=50,xticklabel=\empty,yticklabel=\empty, minor tick num=1,axis lines = middle,xlabel=$y$,ylabel=$x$,label style = {at={(ticklabel cs:1.1)}}]
  \addplot[red,  ultra thick, name path=A] (x, -2 * x * x);
  \addlegendentry{\(-2y^2\)}
  \addplot[blue, ultra thick, name path=B] (x, 1 - 3 * x * x);
  \addlegendentry{\(1-3y^2\)}
  \addplot[gray!40] fill between[of=A and B];
\end{axis}
\end{tikzpicture}
\end{center}

\textbf{Решение.}

$-2y^2 = 1 - 3y^2 \Longleftrightarrow y^2 = 1 \Longleftrightarrow y = \pm 1$, $x(-1) = -2$, $x(1) = -2$

$S_{1} = - \int\limits_{-2}^{0} ((1 - 3y^2) - (-2y^2)) = -\int\limits_{-2}^{0} (1 - y^2) = - (y - \frac{y^3}{3}) \bigg|_{-2}^{0} = (-2 + \frac{8}{3}) = \frac{2}{3}$

$S = 2 S_{1} = \frac{4}{3}$

\hfill

\textbf{Ответ.} $S = \frac{4}{3}$

\subsection{Задание №4}

Нарисовать область, ограниченную линиями, и вычислить ее площадь.

$$
r = 1 + \sqrt{2} \sin \phi
$$

\textbf{График.}

\begin{center}
\begin{tikzpicture}
\begin{polaraxis}[]
\addplot[domain=0:360,samples=300, color=red] {1 + sqrt(2)*sin(x)};
\end{polaraxis}
\end{tikzpicture}
\end{center}

\textbf{Решение.} $S = \frac{1}{2} \int\limits_{\pi/3}^{2\pi/3} (1 + \sqrt{2} \sin \phi) \diff \phi = \frac{1}{2} (\phi - \sqrt{2} \cos \phi) \bigg|_{\pi/3}^{2\pi/3} = \frac{1}{2} ((\frac{2 \pi}{3} + \frac{\sqrt{2}}{2}) - (\frac{\pi}{3} - \frac{\sqrt{2}}{2})) = \frac{1}{2} (\frac{\pi}{3} + \sqrt{2}) = \frac{\pi}{6} + \frac{\sqrt{2}}{2} \approx 1.23$

\hfill

\textbf{Ответ.} $S = 1.23$

\subsection{Задание №6}

Нарисовать дугу кривой и вычислить ее длину.

$$
y = 4 \cosh \frac{x}{4}, \ 0 \le x \le 1
$$

\textbf{График.}

\begin{center}
\begin{tikzpicture}
\begin{axis}[grid=both,xmax=5,ymax=5,xmin=-5,ymin=-5,samples=50,xticklabel=\empty,yticklabel=\empty, minor tick num=1,axis lines = middle,xlabel=$y$,ylabel=$x$,label style = {at={(ticklabel cs:1.1)}}]
  \addplot[red,  ultra thick, domain=-1:1] {4 * cosh(x/4)};
  \addlegendentry{\(4 \cosh \frac{x}{4}\)}
\end{axis}
\end{tikzpicture}
\end{center}

\hfill

\textbf{Решение.} $l = \int\limits_{a}^{b} \sqrt{1 + (f'(x))^2} \diff x = \int\limits_{-1}^{1} \sqrt{1 + (\sinh \frac{x}{4})^2} \diff x = \dots$

$\int \sqrt{1 + \sinh^2 \frac{x}{4}} \diff x = \dots$

Выполним замену $u = \frac{x}{4}$, $\diff u = \frac{1}{4} \diff x$

$\dots = 4 \int \sqrt{1 + \sinh^2 (u)} \diff u = 4 \int \sqrt{\cosh^2 (u)} \diff u = 4 \int \cosh (u) \diff u = 4 \sinh (u) + C = \dots$

Вернемся к переменной $x$

$\dots = 4 \sinh (\frac{x}{4}) + C$

\hfill

$\dots = (4 \sinh (\frac{x}{4})) \bigg|_{-1}^{1} = (4 \sinh \frac{x}{4} - (- 4 \sinh \frac{x}{4})) = 8 \sinh \frac{x}{4}$

\hfill

\textbf{Ответ.} $l = 8 \sinh \frac{x}{4}$

\subsection{Задание №7}

Нарисовать дугу кривой и вычислить ее длину.

$$
y = \frac{t^3}{3}, \ x = t^2, \ -1 \le t \le 1
$$

\textbf{График.}

\begin{center}
\begin{tikzpicture}
\begin{axis}[grid=both,xmax=1,ymax=1,xmin=-1,ymin=-1,samples=50, minor tick num=1,axis lines = middle,xlabel=$y$,ylabel=$x$,label style = {at={(ticklabel cs:1.1)}}]
  \addplot[domain=-1:1, red,  ultra thick, name path=A] (x^2, x^3/3);
\end{axis}
\end{tikzpicture}
\end{center}

\textbf{Решение.}

$l = \int\limits_{t_1}^{t_2} \sqrt{(x'(t))^2 + (y'(t))^2} \diff t = \int\limits_{-1}^{1} \sqrt{(t^2)^2 + (2t)^2} \diff t = \int\limits_{-1}^{1} \sqrt{t^4 + 4t^2} \diff t = 2 \int\limits_{0}^{1} t \sqrt{t^2 + 4} \diff t = \dots$

Выполним замену $u = t^2 + 4$, $\diff u = 2t \diff t$. Получаем новую нижнюю границу $u = 4 + 0^2 = 4$ и новую верхнюю границу $u = 4 + 1^2 = 5$

$\dots = \int\limits_{4}^{5} \sqrt{u} \diff u = \frac{2 u \sqrt{u^2}}{3} \bigg|_{4}^{5} = \frac{2 * 5 \sqrt{5}}{3} - \frac{2 * 4 * \sqrt{4}}{3} = \frac{2}{3} (5 \sqrt{5} - 8)$

\hfill

\textbf{Ответ.} $l = \frac{2}{3} (5\sqrt{5} - 8)$

\subsection{Задание №8}

Нарисовать дугу кривой и вычислить ее длину.

$$
r = \sin \phi + \cos \phi
$$

\textbf{График.}

\begin{center}
\begin{tikzpicture}
\begin{polaraxis}[]
\addplot[domain=0:360,samples=300, color=red] {sin(x) + cos(x)};
\end{polaraxis}
\end{tikzpicture}
\end{center}

\textbf{Решение.}

$l = \int\limits_{\alpha}^{\beta} \sqrt{r^2 (\phi) + (r'(\phi))^2} \diff \phi$

$l = 2 \int\limits_{0}^{\pi} \sqrt{(\sin \phi + \cos \phi)^2 + (\cos \phi - \sin \phi)^2} \diff \phi = 2 \int\limits_{0}^{\pi} \sqrt{(\sin^2 \phi + 2 \sin \phi \cos \phi + \cos^2 \phi) + (\cos^2 \phi - 2 \cos \phi \sin \phi + \sin^2 \phi)} \diff \phi = 2 \int\limits_{0}^{\pi} \sqrt{2 \sin^2 \phi + 2 \cos^2 \phi} \diff \phi = 2 \int\limits_{0}^{\pi} \sqrt{2} \diff \phi = 2 \sqrt{2} \pi$

\hfill

\textbf{Ответ.} $l = 2 \sqrt{2} \pi$

\subsection{Задание №10}

Вычислить несобственный интеграл

$$
\int\limits_{0}^{1/2} \frac{\diff x}{x \ln^2 x}
$$

\textbf{Решение.}


$\int\limits_{0}^{1/2} \frac{\diff x}{x \ln^2 x} = \dots$

\hfill

$\int \frac{\diff x}{x \ln^2 x} = \dots$, выполним замену $u = \ln x$, $\diff u = \frac{1}{x} \diff x$, тогда $\dots = \int \frac{\diff u}{u^2} = - \frac{1}{u}  = \dots$

Вернемся к переменной $x$, $\dots = -\frac{1}{\ln x}$

\hfill

$\dots = \lim\limits_{a \to 0} (-\frac{1}{\ln x}) \bigg|_{a}^{1/2} = \lim\limits_{a \to 0} (-\frac{1}{\ln 1/2} + \frac{1}{\ln a}) = -\frac{1}{\ln 1/2} = \frac{1}{\ln 2}$

\hfill

\textbf{Ответ}: $\int\limits_{0}^{1/2} \frac{\diff x}{x \ln^2 x} = \frac{1}{\ln 2}$

\subsection{Задание №11}

Найти и изобразить в плоскости $x O y$ область определения функции

$$
z = \sqrt{e^{x y} (x - y^2)}
$$

\begin{equation}
  \begin{cases}
    e^{x y} (x - y^2) \ge 0
  \end{cases} \Longrightarrow
  \begin{cases}
    x \ge y^2
  \end{cases} \Longrightarrow
  \begin{cases}
    x \ge 0 \\
    y \ge - \sqrt{x} \\
    y \le \sqrt{x}
  \end{cases}
\end{equation}

\textbf{График.}

\begin{sagesilent}
 var('x, y')
 P = region_plot([x>=0, y>=-sqrt(x), y<=sqrt(x)], (x, -10, 10), (y, -10, 10), plot_points=300)
\end{sagesilent}

\begin{center}
\sageplot[scale=.6]{P}
\end{center}

\end{document}